\documentclass{scrartcl}
\usepackage{pgfplots}
\usepackage{makecell}
\usepackage{multirow} % For merged cells
\usepackage{booktabs} % For better table formatting
\usepackage{calc}
\usepackage{Style_File}
\usepackage{fancyhdr}
\usepackage{array}
\newcolumntype{P}[1]{>{\centering\arraybackslash}p{#1}}
% Recommended preamble:
\usetikzlibrary{arrows.meta}
\usetikzlibrary{backgrounds}
\usepgfplotslibrary{patchplots}
\usepgfplotslibrary{fillbetween}
\pgfplotsset{%
    layers/standard/.define layer set={%
        background,axis background,axis grid,axis ticks,axis lines,axis tick labels,pre main,main,axis descriptions,axis foreground%
    }{
        grid style={/pgfplots/on layer=axis grid},%
        tick style={/pgfplots/on layer=axis ticks},%
        axis line style={/pgfplots/on layer=axis lines},%
        label style={/pgfplots/on layer=axis descriptions},%
        legend style={/pgfplots/on layer=axis descriptions},%
        title style={/pgfplots/on layer=axis descriptions},%
        colorbar style={/pgfplots/on layer=axis descriptions},%
        ticklabel style={/pgfplots/on layer=axis tick labels},%
        axis background@ style={/pgfplots/on layer=axis background},%
        3d box foreground style={/pgfplots/on layer=axis foreground},%
    },
}



\setlength{\headheight}{0.75in}
\setlength{\oddsidemargin}{0in}
        \setlength{\evensidemargin}{0in}
        \setlength{\textwidth}{6.5in}
        \setlength{\headwidth}{7.3in}
        \setlength{\textheight}{8.75in}
        \rfoot{\thepage}
        \renewcommand{\headrulewidth}{0pt} % Remove the header line
        \renewcommand{\footrulewidth}{0pt}
\fancyhead[L,C]{}
\fancyhead[L]{PH3203: Term Project}
\fancyhead[R]{0\thepage}
\usepackage[hidelinks]{hyperref}
\hypersetup{colorlinks=true,linkcolor=cyan!80!black, citecolor=YellowOrange,urlcolor=cyan!80!black}
\fancyhead[C]{ Entanglement Classification using Knots}
\fancyfoot[C]{0\thepage}
\fancyfoot[R,L]{}
\usetikzlibrary{calc}
\pagestyle{fancy}
\renewcommand{\headrulewidth}{0.4pt}

\usepackage{longtable} 
\usepackage[left = 0.7in,
right = 0.7in,
bottom = 0.9in,
top = 0.9in,
a4paper]{geometry}

\title{
        \Huge\textbf{\textcolor{blue}{{Classification of Entanglement using Knots}}} \\[0.5cm]
        \Large\textsc{{PH3203 Term Project} }
}


\author{{\Large Sagnik Seth} \\ \small{22MS026} \and {\Large Jessica  Das} \\ \small{22MS157} \and {\Large Sayan Karmakar} \\ \small{22MS163}}
\date{}
\begin{document}
\makeatletter
\begin{titlepage}
    \begin{tikzpicture}[remember picture, overlay]
        \draw[black, line width=3pt] 
            ($(current page.north west)+(0.6cm,-0.6cm)$) rectangle 
            ($(current page.south east)+(-0.6cm,0.6cm)$);
    \end{tikzpicture}
    \vspace*{2cm}
    \begin{center}
        {\sffamily\bfseries
        \Huge\textbf{\textcolor{blue}{Classification of Entanglement using Knots}}\\[0.5cm] 
        \Large PH3203 Term Project Report \\[0.5cm]
        \Large \textit{Instructor: Prof. Sourin Das}}\\[3cm]\end{center}
       
        \begin{figure}[H]
            \centering
            \includesvg[scale=0.6]{Square_knot.svg}
        \end{figure}
        \vspace{3cm}
        \begin{table}[H]
			\centering
			\def\arraystretch{2}\tabcolsep=1.2cm
			\begin{tabular}{cc}
				\thead{\Large \texttt{Sagnik Seth }                      \\[5pt] \large 22MS026} &
				\thead{\Large  \texttt{Jessica Das }              \\[5pt] \large 22MS157} \\
				\multicolumn{2}{c}{\thead{\Large  \texttt{Sayan Karmakar} \\[5pt] \large 22MS163}}
			\end{tabular}
		\end{table}
\end{titlepage}
\tableofcontents
\newpage
\section{Introduction}
%     Source - Wikipedia

Classifying entanglement is essential because not all quantum states are equally useful for quantum information tasks. Different types of entanglement serve as distinct resources, each suited to specific applications such as quantum algorithms or secure communication protocols like quantum key distribution. Understanding and identifying these entanglement types helps determine how quantum states can be used and manipulated effectively. hi hi hi this is testing\\[0.3cm]
SLOCC (Stochastic Local Operations and Classical Communication) is a method for classifying quantum entanglement. It defines equivalence classes of quantum states based on whether they can be converted into each other using local operations (on individual qubits) and classical communication.\\[0.3cm]
This idea has been used successfully to study three-qubit states, as shown in [1,2], classifying four-qubit states [3–6] Methods have been developed for handling systems with even more qubits [7,8].[0.3cm]
In this paper, the authors have proposed an alternative classification scheme for quantum entanglement based on topological links.\\[0.3cm]
One of the first images that comes to mind when we think of entanglement is that of entangled threads. Naturally, one wonders if we could study quantum entanglement using entangled 'knots'. Aravind~\cite{Aravind1997} was the first to point out the connections between entangled quantum states and classical knot configurations, focussing on similarity between 3-particle GHZ state and Borromean rings. He associated each particle with a ring, entanglement of any set of particles as inability to separate their corresponding rings, and measurement of particle state as cutting its ring. But he noted that performing the measurement in different basis would not lead to the same conclusions. This limit in analogy was dealt with by Sugita~\cite{Sugita2007-ko}. He proposed that cutting the ring is equivalent to tracing out the corresponding particle from the density operator, which is a basis-independent operation. This represents viewing the system as though that particle is no longer present. Moreover, the trace operation helps to generalise the idea to quantum systems with more than 2 levels.  



\section{Classification of Links: A Polynomial Approach}
\subsection{Formalism of the Link Polynomial}
\subsection{Obtaining a Link from an Entangled Quantum State}\label{link_from_state}

\subsection{Obtaining an Entangled Quantum State from a Link}\label{state_from_link}
In this section, we will see how we can obtain a link from a quantum state. This is in general a difficult task to obtain an entangled quantum states from the polynomial as the number of qubits increases. The process in the paper mentions an algorithm which provides an `incomplete' map betwee a given link and a quantum state. Using the procedure, the general structure of the quantum state can be obtained, however, some free coefficients remain which needs to be fixed computationally. Moreover, presently only mixed states satisfying the link can be obtained using this procedure. \\[0.3cm]
Note that although incomplete, the map is still useful since we can ascertain the structure of each state contained in the mixed state. That is, from a possibility of $2^N$ (for $N$ qubits, there are $2^N$ basis states, namely $\ket{0}, \ket{1}, \ldots \ket{2^N-1}$ where each ket is to be assumed in the binary representation) states, we are reducing it to a much smaller number. \\[0.3cm]
For this, we will use the GHZ type of state as a building block which are of the form:
\begin{align*}
    \ket{N^1} = \frac{1}{\sqrt{2}}\brac{\ket{0}^{\otimes N} + \ket{1}^{\otimes N}} \\
\end{align*}
Here $\ket{0}^{\otimes N}$ is the tensor product of $N$ number of $\ket{0}$ states, that is, $\ket{0}^{\otimes N} = \ket{\underbrace{0000\ldots0}_{\mathrm{N\ times}}}$. The state $\ket{N^1}$ is a maximally entangled state of $N$ qubits. The general algorithm to obtain a state from the link is as follows:
\begin{enumerate}
    \item Let a polynomial $P$ be given. Select a term of the given `link' polynomial, say $t$. 
    \item The term $t$ is then mapped to a state of the form $\ket{\mathrm{E_q}}\otimes\ket{\mathrm{S_q}}\otimes\ket{\mathrm{Q_d}}$ where: 
    \begin{itemize}
        \item $\ket{\mathrm{E_q}}$ is the entangled qubit of the GHZ type as specified above, associated to ring variables contained in $t$.
        \item $\ket{\mathrm{S_q}}$ is a separable qubit associated with ring variables not contained in $t$. There are a number of possibilities for this separable qubit and we have to find it computationally.
        \item $\ket{\mathrm{Q_d}}$ is a qudit state which is associated with an artifically introduced ring variable (which is alphabetically the next letter of the largest ring variable present). The states always starts from 0 for the first term and is increased by 1 for each successive term of the polynomial. This will later be traced out, hence is of less significance. 
    \end{itemize}
    \item The full state $\ket{\psi}$ is constructed by summing these individual states obtained for each term of the polynomial. 
    \item The full mixed state characterised by this polynomial is then obtained by tracing out the qudit state $\ket{\mathrm{Q_d}}$.
    $$\boxed{\hat{\rho}(P)=\frac{\Tr_d{\ket{\psi}\bra{\psi}}}{\sqrt{\braket{\psi|\psi}}}}$$
\end{enumerate}
\textbf{Example demonstrating the algorithm:} \\[0.3cm]
We will see a simple example of the algorithm to obtain a state from a link. Consider the polynomial $P(a,b,c) = ab+ac$. This is a three-ring link. 
\begin{itemize}
    \item Let us choose the term $t = ab$. This term has two ring variables thus we will associate a two qubit GHZ type of state to $\ket{\mathrm{E_q}}$. Thus, we have $\ket{\mathrm{E_q}} = \frac{1}{\sqrt{2}}\brac{\ket{00} + \ket{11}}\equiv \ket{2^1}_{ab}$. \\[0.3cm]Since the separable qubit has large possibility, we will denote it generally by $\ket{q_1}$ and this will be associated with the remaining ring variable which is $c$. Thus, $\ket{\mathrm{S_q}} = \ket{q_1}_c$. The remaining term is the qudit state which will be associated to $d$ (since $d$ is alphabetical successor of the largest ring variable $c$). Then we will have the full state:
\begin{align*}
    \ket{\psi_1} &= \ket{2^1}_{ab}\otimes\ket{q_1}_c\otimes\ket{0}_d 
\end{align*}
\item Now, let us choose the next term in the polynomial which is $t = ac$. Similar to above, to the entangled qubit we will associate the two qubit GHZ state, thus, $\ket{\mathrm{E_q}} = \frac{1}{\sqrt{2}}\brac{\ket{00} + \ket{11}}\equiv \ket{2^1}_{ac}$.\\[0.3cm] The separable qubit will be associated with the remaining ring variable $b$ and we will denote it by $\ket{q_2}_b$. The qudit state will be associated with $d$ which is the alphabetical successor of $c$ but this time we will use $\ket{1}_d$ as for each successive term, the qudit state increases to the next level. Thus, we have:
\begin{align*}
    \ket{\psi_2} &= \ket{2^1}_{ac}\otimes\ket{q_2}_b\otimes\ket{1}_d
\end{align*}
\item The full state $\ket{\psi}$ is then obtained by summing the two states obtained above with some coefficients:
\begin{align*}
    \ket{\psi} &= c_1\ket{\psi_1} + c_2\ket{\psi_2} \\
    &=c_1(\ket{2^1}_{ab}\otimes\ket{q_1}_c\otimes\ket{0}_d )+ c_2(\ket{2^1}_{ac}\otimes\ket{q_2}_b\otimes\ket{1}_d)
\end{align*}
Then we can trace out the qudit state $\ket{d}$ to obtain the density matrix of the ring variables:
$$\boxed{\hat{\rho}_{abc}=\frac{\Tr_d{\ket{\psi}\bra{\psi}}}{\sqrt{\braket{\psi|\psi}}}} $$
\end{itemize}
\subsection{Applying to Three Qubit Systems}
As a demonstraction, we will apply our algorithm to three qubit systems. Note that from the rules of the `link' polynomial, the possible basis terms for three qubit system are: $\{ab,ac,bc, abc\}$. Using this, four distinct classes of polynomials are possible:
\begin{align*}
    P_1(a,b,c) &= abc \\
    P_2(a,b,c) &= abc+ab \\
    P_3(a,b,c) &= ab+ac \\
    P_4(a,b,c) &= ab+ac+bc
\end{align*}
\textbf{\large \texttt{3\textsuperscript{1}} Link Class} \\[0.3cm]
Let us start with the $3^1$ link class, which correspond to the Borromean Link. Cutting any of $a,b$ or $c$ will lead to complete separabability and loss of entanglement. 
\begin{figure}[H]
    \centering
  

\tikzset{every picture/.style={line width=0.75pt}} %set default line width to 0.75pt        

\begin{tikzpicture}[x=0.75pt,y=0.75pt,yscale=-1,xscale=1]
%uncomment if require: \path (0,300); %set diagram left start at 0, and has height of 300

%Shape: Arc [id:dp49278097545949884] 
\draw  [draw opacity=0][line width=2.25]  (328.62,163.73) .. controls (321.19,166.44) and (313.18,167.92) .. (304.82,167.92) .. controls (276.06,167.92) and (251.39,150.46) .. (240.8,125.56) -- (304.82,98.33) -- cycle ; \draw  [color={rgb, 255:red, 115; green, 190; blue, 50 }  ,draw opacity=1 ][line width=2.25]  (328.62,163.73) .. controls (321.19,166.44) and (313.18,167.92) .. (304.82,167.92) .. controls (276.06,167.92) and (251.39,150.46) .. (240.8,125.56) ;  
%Shape: Arc [id:dp7313937278600963] 
\draw  [draw opacity=0][line width=2.25]  (380.22,115.26) .. controls (389.75,119.69) and (398.4,126.36) .. (405.34,135.15) .. controls (429.07,165.23) and (423.73,208.87) .. (393.41,232.62) .. controls (363.09,256.37) and (319.27,251.24) .. (295.54,221.15) .. controls (284.6,207.29) and (279.84,190.54) .. (280.85,174.18) -- (350.44,178.15) -- cycle ; \draw  [color={rgb, 255:red, 15; green, 0; blue, 255 }  ,draw opacity=1 ][line width=2.25]  (380.22,115.26) .. controls (389.75,119.69) and (398.4,126.36) .. (405.34,135.15) .. controls (429.07,165.23) and (423.73,208.87) .. (393.41,232.62) .. controls (363.09,256.37) and (319.27,251.24) .. (295.54,221.15) .. controls (284.6,207.29) and (279.84,190.54) .. (280.85,174.18) ;  
%Shape: Arc [id:dp9082888658781946] 
\draw  [draw opacity=0][line width=2.25]  (235.71,106.58) .. controls (232.15,77.58) and (247.26,48.37) .. (275.16,35.34) .. controls (309.93,19.08) and (351.4,34.11) .. (367.78,68.91) .. controls (383.06,101.37) and (371.11,139.56) .. (341.2,157.73) -- (304.82,98.33) -- cycle ; \draw  [color={rgb, 255:red, 115; green, 190; blue, 50 }  ,draw opacity=1 ][line width=2.25]  (235.71,106.58) .. controls (232.15,77.58) and (247.26,48.37) .. (275.16,35.34) .. controls (309.93,19.08) and (351.4,34.11) .. (367.78,68.91) .. controls (383.06,101.37) and (371.11,139.56) .. (341.2,157.73) ;  
%Shape: Arc [id:dp7357603003827937] 
\draw  [draw opacity=0][line width=2.25]  (302.44,238.64) .. controls (269.08,257.52) and (226.71,245.99) .. (207.75,212.86) .. controls (188.77,179.69) and (200.44,137.44) .. (233.81,118.48) .. controls (255.62,106.09) and (281.32,106.72) .. (301.8,117.92) -- (268.17,178.53) -- cycle ; \draw  [color={rgb, 255:red, 255; green, 8; blue, 8 }  ,draw opacity=1 ][line width=2.25]  (302.44,238.64) .. controls (269.08,257.52) and (226.71,245.99) .. (207.75,212.86) .. controls (188.77,179.69) and (200.44,137.44) .. (233.81,118.48) .. controls (255.62,106.09) and (281.32,106.72) .. (301.8,117.92) ;  
%Shape: Arc [id:dp02993817280793909] 
\draw  [draw opacity=0][line width=2.25]  (283.51,159.01) .. controls (286.66,148.19) and (292.5,137.97) .. (301.03,129.39) .. controls (318.8,111.52) and (343.85,105.22) .. (366.69,110.55) -- (350.44,178.15) -- cycle ; \draw  [color={rgb, 255:red, 15; green, 0; blue, 255 }  ,draw opacity=1 ][line width=2.25]  (283.51,159.01) .. controls (286.66,148.19) and (292.5,137.97) .. (301.03,129.39) .. controls (318.8,111.52) and (343.85,105.22) .. (366.69,110.55) ;  
%Shape: Arc [id:dp6464277844327507] 
\draw  [draw opacity=0][line width=2.25]  (315.85,128.16) .. controls (331.94,143.37) and (340.46,165.99) .. (336.68,189.48) .. controls (334.06,205.76) and (325.96,219.8) .. (314.56,230.03) -- (268.17,178.53) -- cycle ; \draw  [color={rgb, 255:red, 255; green, 8; blue, 8 }  ,draw opacity=1 ][line width=2.25]  (315.85,128.16) .. controls (331.94,143.37) and (340.46,165.99) .. (336.68,189.48) .. controls (334.06,205.76) and (325.96,219.8) .. (314.56,230.03) ;  




\end{tikzpicture}
  
  \caption{The Borromean link, characterising the $3^1$ link class.}
\end{figure}
\noindent
We already know that on its own the GHZ state characterises the $3^1$ link, as discussed in the preceding works. Thus, we have the pure state:
$$\ket{3^1}_{abc} = \frac{1}{\sqrt{2}}\brac{\ket{000}_{abc}+\ket{111}_{abc}}$$\\[0.3cm]
The density matrix corresponding to the state is found to be:
\begin{equation}
    \rho_{abc} =
    \left[
    \begin{array}{cccccccc}
    0.5 & 0.0 & 0.0 & 0.0 & 0.0 & 0.0 & 0.0 & 0.5 \\
    0.0 & 0.0 & 0.0 & 0.0 & 0.0 & 0.0 & 0.0 & 0.0 \\
    0.0 & 0.0 & 0.0 & 0.0 & 0.0 & 0.0 & 0.0 & 0.0 \\
    0.0 & 0.0 & 0.0 & 0.0 & 0.0 & 0.0 & 0.0 & 0.0 \\
    0.0 & 0.0 & 0.0 & 0.0 & 0.0 & 0.0 & 0.0 & 0.0 \\
    0.0 & 0.0 & 0.0 & 0.0 & 0.0 & 0.0 & 0.0 & 0.0 \\
    0.0 & 0.0 & 0.0 & 0.0 & 0.0 & 0.0 & 0.0 & 0.0 \\
    0.5 & 0.0 & 0.0 & 0.0 & 0.0 & 0.0 & 0.0 & 0.5 \\
    \end{array}
    \right]
    \end{equation}
    The partial transpose with respect to any of the subsystem (since the polynomial is symmetric) is same and is given by:
    \begin{equation}
        \rho^{T_a/T_b/T_c}_{abc} =
        \left[
        \begin{array}{cccccccc}
        0.5 & 0.0 & 0.0 & 0.0 & 0.0 & 0.0 & 0.0 & 0.0 \\
        0.0 & 0.0 & 0.0 & 0.0 & 0.0 & 0.0 & 0.0 & 0.0 \\
        0.0 & 0.0 & 0.0 & 0.0 & 0.0 & 0.0 & 0.0 & 0.0 \\
        0.0 & 0.0 & 0.0 & 0.0 & 0.5 & 0.0 & 0.0 & 0.0 \\
        0.0 & 0.0 & 0.0 & 0.5 & 0.0 & 0.0 & 0.0 & 0.0 \\
        0.0 & 0.0 & 0.0 & 0.0 & 0.0 & 0.0 & 0.0 & 0.0 \\
        0.0 & 0.0 & 0.0 & 0.0 & 0.0 & 0.0 & 0.0 & 0.0 \\
        0.0 & 0.0 & 0.0 & 0.0 & 0.0 & 0.0 & 0.0 & 0.5 \\
        \end{array}
        \right]
        \end{equation}
        The eigenvalues corresponding to this matrix are 0.0, 0.5 and -0.5. Since there are negative eigenvalues, we can conclude that the system exhibits \textbf{tripartite entanglement} as a whole.\\[0.3cm]
        Now, let us reduce the system by tracing out one of the variable. Since the polynomial is symmetric, we can choose any of the variable, say $c$. The reduced density matrix is given by:
        \begin{equation}
            \rho_{ab}
            \left[
            \begin{array}{cccc}
            0.5 & 0.0 & 0.0 & 0.0 \\
            0.0 & 0.0 & 0.0 & 0.0 \\
            0.0 & 0.0 & 0.0 & 0.0 \\
            0.0 & 0.0 & 0.0 & 0.5 \\
            \end{array}
            \right]
            \end{equation} 
            The partial transpose with respect to $a$ or $b$ results in the same above matrix which have eigenvalues $0.5$ and $0.0$ which are all positive, thus concluding the absence of any entanglement in the system. This is consistent with the fact that for the Borromean link, cutting any link with result in complete separability of the links. \\[0.3cm]
            \textbf{\large \texttt{3\textsuperscript{2}} Link Class} \\[0.3cm]
Let us now consider the $3^2$ link class given by $P_2(a,b,c) = abc+ab$. Cutting any of $a,b$ will lead to complete separabality but if we cut $c$, then the other rings will remain entangled. The link can be represented as: 
\begin{figure}[H]
    \centering
  

\tikzset{every picture/.style={line width=0.75pt}} %set default line width to 0.75pt        

\begin{tikzpicture}[x=0.75pt,y=0.75pt,yscale=-1,xscale=1]
%uncomment if require: \path (0,300); %set diagram left start at 0, and has height of 300

%Curve Lines [id:da09943248726534226] 
\draw [color={rgb, 255:red, 255; green, 8; blue, 8 }  ,draw opacity=1 ][line width=2.25]    (343,46.68) .. controls (407,41.68) and (460,141.77) .. (337,197.77) ;
%Curve Lines [id:da6322873360352863] 
\draw [color={rgb, 255:red, 255; green, 8; blue, 8 }  ,draw opacity=1 ][line width=2.25]    (322,203.75) .. controls (309.03,209.29) and (311.19,209.93) .. (302.21,215.27) .. controls (293.23,220.61) and (231,265.75) .. (297,280.92) ;
%Curve Lines [id:da5011644671112231] 
\draw [color={rgb, 255:red, 255; green, 8; blue, 8 }  ,draw opacity=1 ][line width=2.25]    (331,46.57) .. controls (216,25.57) and (214,74.97) .. (215,104.68) .. controls (216,134.4) and (245,168.77) .. (301,202.77) ;
%Curve Lines [id:da44718777975671287] 
\draw [color={rgb, 255:red, 255; green, 8; blue, 8 }  ,draw opacity=1 ][line width=2.25]    (328,240.75) .. controls (352,266.75) and (326.73,282.41) .. (297,280.92) ;
%Shape: Arc [id:dp40054047053808206] 
\draw  [draw opacity=0][line width=2.25]  (321.85,192.58) .. controls (331.21,200.68) and (334.92,214.11) .. (330.11,226.28) .. controls (324.03,241.69) and (306.61,249.26) .. (291.19,243.17) .. controls (287.84,241.85) and (284.86,239.99) .. (282.31,237.73) -- (302.21,215.27) -- cycle ; \draw  [color={rgb, 255:red, 115; green, 190; blue, 50 }  ,draw opacity=1 ][line width=2.25]  (321.85,192.58) .. controls (331.21,200.68) and (334.92,214.11) .. (330.11,226.28) .. controls (324.03,241.69) and (306.61,249.26) .. (291.19,243.17) .. controls (287.84,241.85) and (284.86,239.99) .. (282.31,237.73) ;  
%Shape: Arc [id:dp11594177788693838] 
\draw  [draw opacity=0][line width=2.25]  (275.74,229.39) .. controls (271.75,221.9) and (270.94,212.77) .. (274.3,204.25) .. controls (275.71,200.7) and (277.71,197.56) .. (280.16,194.91) -- (302.21,215.27) -- cycle ; \draw  [color={rgb, 255:red, 115; green, 190; blue, 50 }  ,draw opacity=1 ][line width=2.25]  (275.74,229.39) .. controls (271.75,221.9) and (270.94,212.77) .. (274.3,204.25) .. controls (275.71,200.7) and (277.71,197.56) .. (280.16,194.91) ;  
%Straight Lines [id:da5976132825673692] 
\draw [color={rgb, 255:red, 255; green, 8; blue, 8 }  ,draw opacity=1 ][fill={rgb, 255:red, 208; green, 2; blue, 27 }  ,fill opacity=1 ][line width=2.25]    (311,215.75) -- (323,232.75) ;
%Curve Lines [id:da5325527173421334] 
\draw [color={rgb, 255:red, 115; green, 190; blue, 50 }  ,draw opacity=1 ][line width=2.25]    (335,79.68) .. controls (335,114.68) and (300,151.68) .. (321.85,192.58) ;
%Curve Lines [id:da2732197209903964] 
\draw [color={rgb, 255:red, 115; green, 190; blue, 50 }  ,draw opacity=1 ][line width=2.25]    (277,82.3) .. controls (277,117.3) and (316,147.68) .. (285.85,186.58) ;
%Curve Lines [id:da21608820690853647] 
\draw [color={rgb, 255:red, 115; green, 190; blue, 50 }  ,draw opacity=1 ][line width=2.25]    (278,73.65) .. controls (286,47.65) and (274,80.12) .. (288,47.12) ;
%Curve Lines [id:da1149451703987957] 
\draw [color={rgb, 255:red, 115; green, 190; blue, 50 }  ,draw opacity=1 ][line width=2.25]    (335,71.12) .. controls (343,47.12) and (328,-29.7) .. (293,36.3) ;
%Curve Lines [id:da6104443808138464] 
\draw [color={rgb, 255:red, 15; green, 0; blue, 255 }  ,draw opacity=1 ][line width=2.25]    (272,57.3) .. controls (200,86.3) and (405,88.3) .. (340,56.3) ;
%Curve Lines [id:da9503680162575591] 
\draw [color={rgb, 255:red, 15; green, 0; blue, 255 }  ,draw opacity=1 ][line width=2.25]    (289,54.8) .. controls (296,52.68) and (313,51.68) .. (332,54.68) ;




\end{tikzpicture}
  
  \caption{The knot diagram, characterising the $3^2$ link class.}
\end{figure}
\noindent
Here the blue knot corresponds to $c$ while the other two correspond to $a$ and $b$ (a,b are symmetric in the polynomial). An example of a pure state is found, as mentioned in the paper: 
$$\ket{3^2}_{abc} = \frac{1}{\sqrt{3}}\brac{\ket{000}_{abc} + \ket{111}_{abc} + \ket{001}_{abc}} $$
To check that this state indeed statistfies the link, let us calculate the density operator $\hat{\rho}_{abc}=\ket{3^2}_{abc}\bra{3^2}_{abc}$ and then check for the PPT test for each cuts. 
The density matrix obtained is: 
\begin{equation}
    \rho_{abc}=
    \left[
    \begin{array}{cccccccc}
    0.333 & 0.333 & 0.0 & 0.0 & 0.0 & 0.0 & 0.0 & 0.333 \\
    0.333 & 0.333 & 0.0 & 0.0 & 0.0 & 0.0 & 0.0 & 0.333 \\
    0.0 & 0.0 & 0.0 & 0.0 & 0.0 & 0.0 & 0.0 & 0.0 \\
    0.0 & 0.0 & 0.0 & 0.0 & 0.0 & 0.0 & 0.0 & 0.0 \\
    0.0 & 0.0 & 0.0 & 0.0 & 0.0 & 0.0 & 0.0 & 0.0 \\
    0.0 & 0.0 & 0.0 & 0.0 & 0.0 & 0.0 & 0.0 & 0.0 \\
    0.0 & 0.0 & 0.0 & 0.0 & 0.0 & 0.0 & 0.0 & 0.0 \\
    0.333 & 0.333 & 0.0 & 0.0 & 0.0 & 0.0 & 0.0 & 0.333 \\
    \end{array}
    \right]
    \end{equation}
    Since the variables $a$ and $b$ are symmetrix, we can choose to analyse only one of them and $c$. We then see the partial transpose with respect to $a$ and $c$. The matrices are given by:
    \begin{align}
        \rho^{T_a}_{abc} &=
        \left[
        \begin{array}{cccccccc}
        0.333 & 0.333 & 0.0 & 0.0 & 0.0 & 0.0 & 0.0 & 0.0 \\
        0.333 & 0.333 & 0.0 & 0.0 & 0.0 & 0.0 & 0.0 & 0.0 \\
        0.0 & 0.0 & 0.0 & 0.0 & 0.0 & 0.0 & 0.0 & 0.0 \\
        0.0 & 0.0 & 0.0 & 0.0 & 0.333 & 0.333 & 0.0 & 0.0 \\
        0.0 & 0.0 & 0.0 & 0.333 & 0.0 & 0.0 & 0.0 & 0.0 \\
        0.0 & 0.0 & 0.0 & 0.333 & 0.0 & 0.0 & 0.0 & 0.0 \\
        0.0 & 0.0 & 0.0 & 0.0 & 0.0 & 0.0 & 0.0 & 0.0 \\
        0.0 & 0.0 & 0.0 & 0.0 & 0.0 & 0.0 & 0.0 & 0.333 \\
        \end{array}
        \right]\\
            \rho^{T_c}_{abc} &=
            \left[
            \begin{array}{cccccccc}
            0.333 & 0.333 & 0.0 & 0.0 & 0.0 & 0.0 & 0.0 & 0.0 \\
            0.333 & 0.333 & 0.0 & 0.0 & 0.0 & 0.0 & 0.333 & 0.333 \\
            0.0 & 0.0 & 0.0 & 0.0 & 0.0 & 0.0 & 0.0 & 0.0 \\
            0.0 & 0.0 & 0.0 & 0.0 & 0.0 & 0.0 & 0.0 & 0.0 \\
            0.0 & 0.0 & 0.0 & 0.0 & 0.0 & 0.0 & 0.0 & 0.0 \\
            0.0 & 0.0 & 0.0 & 0.0 & 0.0 & 0.0 & 0.0 & 0.0 \\
            0.0 & 0.333 & 0.0 & 0.0 & 0.0 & 0.0 & 0.0 & 0.0 \\
            0.0 & 0.333 & 0.0 & 0.0 & 0.0 & 0.0 & 0.0 & 0.333 \\
            \end{array}
            \right]
            \end{align}
            The above matrices have eigenvalues $\lambda_a =-0.471, 0.0, 0.333, 0.471, 0.666$ and $\lambda_c=-0.333, 0.0, 0.127, 0.333, 0.872 $ respectively. Since there are negative eigenvalues, we can conclude that the system exhibits \textbf{tripartite entanglement} as a whole.\\[0.3cm]
            The reduced density matrices are given by:
            \begin{align}
                \rho_{bc} &=
                \left[
                \begin{array}{cccc}
                0.333 & 0.333 & 0.0 & 0.0 \\
                0.333 & 0.333 & 0.0 & 0.0 \\
                0.0 & 0.0 & 0.0 & 0.0 \\
                0.0 & 0.0 & 0.0 & 0.333 \\
                \end{array}
                \right]\\
                \rho_{ab} &=
                    \left[
                    \begin{array}{cccc}
                    0.666 & 0.0 & 0.0 & 0.333 \\
                    0.0 & 0.0 & 0.0 & 0.0 \\
                    0.0 & 0.0 & 0.0 & 0.0 \\
                    0.333 & 0.0 & 0.0 & 0.333 \\
                    \end{array}
                    \right]
                    \end{align}
                   We now obtain the partial tranpose for the PPT test. We note that $\rho_{bc}^{T_b/T_c}$ is the same as that of the above matrix $\rho_{bc}$ whose eigenvalues are $0.0, 0.333, 0.666$ which are all positive. Thus, we can conclude that the system is separable. On the other hand, we obtain:
                   \begin{equation}
                    \rho_{ab}^{T_a} =
                    \left[
                    \begin{array}{cccc}
                    0.666 & 0.0 & 0.0 & 0.0 \\
                    0.0 & 0.0 & 0.333 & 0.0 \\
                    0.0 & 0.333 & 0.0 & 0.0 \\
                    0.0 & 0.0 & 0.0 & 0.333 \\
                    \end{array}
                    \right]
                    \end{equation}
                    The eigenvalues of this matrix are $0.333, -0.333, 0.666$, one of which is negative, thus denoting the presence of entanglement. This successfully verifies the behaviour of the link $abc+ab$. \\[0.3cm]
Using the above algorithm, we can also construct the mixed state corresponding to the link by considering the state:
$$\ket{\psi_2} = \ket{3^1}_{abc}\ket{0}_d + \ket{2^1}_{ab}\ket{0}_c\ket{1}_d$$
It is to be noted that this class has not been described in the previous works \cite{Aravind1997,Sugita2007-ko}. The density operator is then given by:
$$\hat{\rho}(a,b,c) = \frac{\Tr_d\ket{\psi_2}\bra{\psi_2}}{\sqrt{\braket{\psi_2|\psi_2}}} $$
Using numerical calculation and tracing out subsystem $d$, we found the density matrix to be:
\begin{equation}
    \rho_{abc}=
    \left[
    \begin{array}{cccccccc}
    0.5 & 0.0 & 0.0 & 0.0 & 0.0 & 0.0 & 0.25 & 0.25 \\
    0.0 & 0.0 & 0.0 & 0.0 & 0.0 & 0.0 & 0.0 & 0.0 \\
    0.0 & 0.0 & 0.0 & 0.0 & 0.0 & 0.0 & 0.0 & 0.0 \\
    0.0 & 0.0 & 0.0 & 0.0 & 0.0 & 0.0 & 0.0 & 0.0 \\
    0.0 & 0.0 & 0.0 & 0.0 & 0.0 & 0.0 & 0.0 & 0.0 \\
    0.0 & 0.0 & 0.0 & 0.0 & 0.0 & 0.0 & 0.0 & 0.0 \\
    0.25 & 0.0 & 0.0 & 0.0 & 0.0 & 0.0 & 0.25 & 0.0 \\
    0.25 & 0.0 & 0.0 & 0.0 & 0.0 & 0.0 & 0.0 & 0.25 \\
    \end{array}
    \right]
    \end{equation}
\textbf{\large \texttt{3\textsuperscript{3}} Link Class} \\[0.3cm]
We now analyse the link class given by $P_3(a,b,c) = ab+ac$. The link diagram is given by:
\begin{figure}[H]
    \centering
  

\tikzset{every picture/.style={line width=0.75pt}} %set default line width to 0.75pt        

\begin{tikzpicture}[x=0.75pt,y=0.75pt,yscale=-1,xscale=1]
%uncomment if require: \path (0,300); %set diagram left start at 0, and has height of 300

%Shape: Arc [id:dp45545170377512756] 
\draw  [draw opacity=0][line width=2.25]  (234.67,185.2) .. controls (225.14,172.55) and (219.44,156.46) .. (219.44,138.93) .. controls (219.44,98.21) and (250.25,65.2) .. (288.26,65.2) .. controls (326.26,65.2) and (357.08,98.21) .. (357.08,138.93) .. controls (357.08,155.77) and (351.81,171.29) .. (342.94,183.71) -- (288.26,138.93) -- cycle ; \draw  [color={rgb, 255:red, 15; green, 0; blue, 255 }  ,draw opacity=1 ][line width=2.25]  (234.67,185.2) .. controls (225.14,172.55) and (219.44,156.46) .. (219.44,138.93) .. controls (219.44,98.21) and (250.25,65.2) .. (288.26,65.2) .. controls (326.26,65.2) and (357.08,98.21) .. (357.08,138.93) .. controls (357.08,155.77) and (351.81,171.29) .. (342.94,183.71) ;  
%Shape: Arc [id:dp43179396468586795] 
\draw  [draw opacity=0][line width=2.25]  (336.36,192.63) .. controls (324.26,205.05) and (307.81,212.67) .. (289.56,212.67) .. controls (271.41,212.67) and (254.78,205.14) .. (242.27,192.85) -- (288.26,138.93) -- cycle ; \draw  [color={rgb, 255:red, 15; green, 0; blue, 255 }  ,draw opacity=1 ][line width=2.25]  (336.36,192.63) .. controls (324.26,205.05) and (307.81,212.67) .. (289.56,212.67) .. controls (271.41,212.67) and (254.78,205.14) .. (242.27,192.85) ;  
%Shape: Arc [id:dp1272193908857524] 
\draw  [draw opacity=0][line width=2.25]  (241.85,92.84) .. controls (250.99,105.27) and (256.56,120.84) .. (256.86,137.73) .. controls (257.58,178.46) and (227.35,211.47) .. (189.34,211.47) .. controls (151.34,211.47) and (119.94,178.46) .. (119.23,137.73) .. controls (118.51,97.01) and (148.74,64) .. (186.74,64) .. controls (203.64,64) and (219.23,70.53) .. (231.4,81.35) -- (188.04,137.73) -- cycle ; \draw  [color={rgb, 255:red, 255; green, 8; blue, 8 }  ,draw opacity=1 ][line width=2.25]  (241.85,92.84) .. controls (250.99,105.27) and (256.56,120.84) .. (256.86,137.73) .. controls (257.58,178.46) and (227.35,211.47) .. (189.34,211.47) .. controls (151.34,211.47) and (119.94,178.46) .. (119.23,137.73) .. controls (118.51,97.01) and (148.74,64) .. (186.74,64) .. controls (203.64,64) and (219.23,70.53) .. (231.4,81.35) ;  
%Shape: Arc [id:dp33396414047455336] 
\draw  [draw opacity=0][line width=2.25]  (335.01,96.98) .. controls (327.1,108.89) and (322.33,123.35) .. (322.06,138.93) .. controls (321.34,179.65) and (351.57,212.67) .. (389.58,212.67) .. controls (427.59,212.67) and (458.99,179.65) .. (459.71,138.93) .. controls (460.43,98.21) and (430.19,65.2) .. (392.18,65.2) .. controls (373.16,65.2) and (355.8,73.46) .. (343.11,86.82) -- (390.88,138.93) -- cycle ; \draw  [color={rgb, 255:red, 115; green, 190; blue, 50 }  ,draw opacity=1 ][line width=2.25]  (335.01,96.98) .. controls (327.1,108.89) and (322.33,123.35) .. (322.06,138.93) .. controls (321.34,179.65) and (351.57,212.67) .. (389.58,212.67) .. controls (427.59,212.67) and (458.99,179.65) .. (459.71,138.93) .. controls (460.43,98.21) and (430.19,65.2) .. (392.18,65.2) .. controls (373.16,65.2) and (355.8,73.46) .. (343.11,86.82) ;  




\end{tikzpicture}
  
  \caption{The knot diagram, characterising the $3^3$ link class.}
\end{figure}
\noindent
We can see that the knot polynomial is symmetric in $b$ and $c$. Thus, if we cut either $b$ or $c$, the other will remain entangled with $a$ but complete separation results from cutting $a$. This case has already been discussed in section \ref{link_from_state} and a pure state has been obtained. The mixed state can be obtained using the state:
$$\ket{\psi_3} = \ket{2^1}_{ab}\ket{0}_c\ket{0}_d + \ket{2^1}_{ac}\ket{1}_b\ket{1}_d  $$ 
The density matrix obtained for the system is: 
\begin{equation}
    \rho_{abc}=
    \left[
    \begin{array}{cccccccc}
    0.25 & 0.0 & 0.0 & 0.0 & 0.0 & 0.0 & 0.25 & 0.0 \\
    0.0 & 0.0 & 0.0 & 0.0 & 0.0 & 0.0 & 0.0 & 0.0 \\
    0.0 & 0.0 & 0.25 & 0.0 & 0.0 & 0.0 & 0.0 & 0.25 \\
    0.0 & 0.0 & 0.0 & 0.0 & 0.0 & 0.0 & 0.0 & 0.0 \\
    0.0 & 0.0 & 0.0 & 0.0 & 0.0 & 0.0 & 0.0 & 0.0 \\
    0.0 & 0.0 & 0.0 & 0.0 & 0.0 & 0.0 & 0.0 & 0.0 \\
    0.25 & 0.0 & 0.0 & 0.0 & 0.0 & 0.0 & 0.25 & 0.0 \\
    0.0 & 0.0 & 0.25 & 0.0 & 0.0 & 0.0 & 0.0 & 0.25 \\
    \end{array}
    \right]
    \end{equation}
\textbf{\large \texttt{3\textsuperscript{4}} Link Class} \\[0.3cm]
Let us now analyse the class $P_4(a,b,c) = ab+ac+bc$. The link diagram is given by:
\begin{figure}[H]
    \centering
  

\tikzset{every picture/.style={line width=0.75pt}} %set default line width to 0.75pt        

\begin{tikzpicture}[x=0.75pt,y=0.75pt,yscale=-1,xscale=1]
%uncomment if require: \path (0,300); %set diagram left start at 0, and has height of 300

%Shape: Arc [id:dp5841759180684423] 
\draw  [draw opacity=0][line width=2.25]  (290.27,165.55) .. controls (273.28,157.84) and (259.7,143.9) .. (252.41,126.63) -- (320.87,97.51) -- cycle ; \draw  [color={rgb, 255:red, 115; green, 190; blue, 50 }  ,draw opacity=1 ][line width=2.25]  (290.27,165.55) .. controls (273.28,157.84) and (259.7,143.9) .. (252.41,126.63) ;  
%Shape: Arc [id:dp7577998399786959] 
\draw  [draw opacity=0][line width=2.25]  (401.49,115.68) .. controls (411.67,120.44) and (420.91,127.58) .. (428.32,137) .. controls (453.68,169.27) and (447.97,216.08) .. (415.56,241.55) .. controls (383.16,267.03) and (336.33,261.52) .. (310.96,229.25) .. controls (287.54,199.45) and (290.62,157.26) .. (316.74,130.91) -- (369.64,183.13) -- cycle ; \draw  [color={rgb, 255:red, 15; green, 0; blue, 255 }  ,draw opacity=1 ][line width=2.25]  (401.49,115.68) .. controls (411.67,120.44) and (420.91,127.58) .. (428.32,137) .. controls (453.68,169.27) and (447.97,216.08) .. (415.56,241.55) .. controls (383.16,267.03) and (336.33,261.52) .. (310.96,229.25) .. controls (287.54,199.45) and (290.62,157.26) .. (316.74,130.91) ;  
%Shape: Arc [id:dp896095108426711] 
\draw  [draw opacity=0][line width=2.25]  (247.58,110.23) .. controls (241.88,77.87) and (258.12,44.5) .. (289.17,29.93) .. controls (326.35,12.5) and (370.67,28.62) .. (388.18,65.94) .. controls (405.69,103.27) and (389.75,147.66) .. (352.57,165.1) .. controls (336.02,172.86) and (318.05,173.97) .. (301.66,169.52) -- (320.87,97.51) -- cycle ; \draw  [color={rgb, 255:red, 115; green, 190; blue, 50 }  ,draw opacity=1 ][line width=2.25]  (247.58,110.23) .. controls (241.88,77.87) and (258.12,44.5) .. (289.17,29.93) .. controls (326.35,12.5) and (370.67,28.62) .. (388.18,65.94) .. controls (405.69,103.27) and (389.75,147.66) .. (352.57,165.1) .. controls (336.02,172.86) and (318.05,173.97) .. (301.66,169.52) ;  
%Shape: Arc [id:dp5765147440419169] 
\draw  [draw opacity=0][line width=2.25]  (318.33,248.02) .. controls (282.68,268.27) and (237.38,255.9) .. (217.12,220.36) .. controls (196.83,184.78) and (209.3,139.45) .. (244.97,119.12) .. controls (280.64,98.78) and (326,111.14) .. (346.28,146.72) .. controls (347.96,149.66) and (349.41,152.67) .. (350.65,155.72) -- (281.7,183.54) -- cycle ; \draw  [color={rgb, 255:red, 255; green, 8; blue, 8 }  ,draw opacity=1 ][line width=2.25]  (318.33,248.02) .. controls (282.68,268.27) and (237.38,255.9) .. (217.12,220.36) .. controls (196.83,184.78) and (209.3,139.45) .. (244.97,119.12) .. controls (280.64,98.78) and (326,111.14) .. (346.28,146.72) .. controls (347.96,149.66) and (349.41,152.67) .. (350.65,155.72) ;  
%Shape: Arc [id:dp9828413593386676] 
\draw  [draw opacity=0][line width=2.25]  (330.95,119.62) .. controls (347.94,109.22) and (368.18,106.21) .. (387.01,110.62) -- (369.64,183.13) -- cycle ; \draw  [color={rgb, 255:red, 15; green, 0; blue, 255 }  ,draw opacity=1 ][line width=2.25]  (330.95,119.62) .. controls (347.94,109.22) and (368.18,106.21) .. (387.01,110.62) ;  
%Shape: Arc [id:dp23316225635631627] 
\draw  [draw opacity=0][line width=2.25]  (355.12,172.79) .. controls (356.17,180.07) and (356.16,187.63) .. (354.93,195.28) .. controls (352.14,212.7) and (343.51,227.73) .. (331.38,238.69) -- (281.7,183.54) -- cycle ; \draw  [color={rgb, 255:red, 255; green, 8; blue, 8 }  ,draw opacity=1 ][line width=2.25]  (355.12,172.79) .. controls (356.17,180.07) and (356.16,187.63) .. (354.93,195.28) .. controls (352.14,212.7) and (343.51,227.73) .. (331.38,238.69) ;  




\end{tikzpicture}
  
  \caption{The knot diagram, characterising the $3^4$ link class.}
\end{figure}
\noindent
This link demostrates a behaviour opposite to that of the Borromean ring, that is, setting any one variable to zero with not result in complete separability. The other two links remain entangled. The pure state having the characteristic of the link is the \textbf{W state} which has been documented in previous works:
$$\ket{3^4} = \frac{1}{\sqrt{3}}\brac{\ket{001}_{abc}+\ket{010}_{abc}+\ket{100}_{abc}}$$ 
Let us check whether this state satisfies the link property. The density matrix is given by:
\begin{equation}
    \rho{abc}=
    \left[
    \begin{array}{cccccccc}
    0.0 & 0.0 & 0.0 & 0.0 & 0.0 & 0.0 & 0.0 & 0.0 \\
    0.0 & 0.333 & 0.333 & 0.0 & 0.333 & 0.0 & 0.0 & 0.0 \\
    0.0 & 0.333 & 0.333 & 0.0 & 0.333 & 0.0 & 0.0 & 0.0 \\
    0.0 & 0.0 & 0.0 & 0.0 & 0.0 & 0.0 & 0.0 & 0.0 \\
    0.0 & 0.333 & 0.333 & 0.0 & 0.333 & 0.0 & 0.0 & 0.0 \\
    0.0 & 0.0 & 0.0 & 0.0 & 0.0 & 0.0 & 0.0 & 0.0 \\
    0.0 & 0.0 & 0.0 & 0.0 & 0.0 & 0.0 & 0.0 & 0.0 \\
    0.0 & 0.0 & 0.0 & 0.0 & 0.0 & 0.0 & 0.0 & 0.0 \\
    \end{array}
    \right]
    \end{equation}
    The corresponding partial transpose with respect to the three variables are given by:
    \begin{equation}
        \rho^{T_a}_{abc} =
        \left[
        \begin{array}{cccccccc}
        0.0 & 0.0 & 0.0 & 0.0 & 0.0 & 0.333 & 0.333 & 0.0 \\
        0.0 & 0.333 & 0.333 & 0.0 & 0.0 & 0.0 & 0.0 & 0.0 \\
        0.0 & 0.333 & 0.333 & 0.0 & 0.0 & 0.0 & 0.0 & 0.0 \\
        0.0 & 0.0 & 0.0 & 0.0 & 0.0 & 0.0 & 0.0 & 0.0 \\
        0.0 & 0.0 & 0.0 & 0.0 & 0.333 & 0.0 & 0.0 & 0.0 \\
        0.333 & 0.0 & 0.0 & 0.0 & 0.0 & 0.0 & 0.0 & 0.0 \\
        0.333 & 0.0 & 0.0 & 0.0 & 0.0 & 0.0 & 0.0 & 0.0 \\
        0.0 & 0.0 & 0.0 & 0.0 & 0.0 & 0.0 & 0.0 & 0.0 \\
        \end{array}
        \right]
        \end{equation}
        \begin{equation}
            \rho^{T_b}_{abc}=
            \left[
            \begin{array}{cccccccc}
            0.0 & 0.0 & 0.0 & 0.333 & 0.0 & 0.0 & 0.333 & 0.0 \\
            0.0 & 0.333 & 0.0 & 0.0 & 0.333 & 0.0 & 0.0 & 0.0 \\
            0.0 & 0.0 & 0.333 & 0.0 & 0.0 & 0.0 & 0.0 & 0.0 \\
            0.333 & 0.0 & 0.0 & 0.0 & 0.0 & 0.0 & 0.0 & 0.0 \\
            0.0 & 0.333 & 0.0 & 0.0 & 0.333 & 0.0 & 0.0 & 0.0 \\
            0.0 & 0.0 & 0.0 & 0.0 & 0.0 & 0.0 & 0.0 & 0.0 \\
            0.333 & 0.0 & 0.0 & 0.0 & 0.0 & 0.0 & 0.0 & 0.0 \\
            0.0 & 0.0 & 0.0 & 0.0 & 0.0 & 0.0 & 0.0 & 0.0 \\
            \end{array}
            \right]
            \end{equation}
            \begin{equation}
                \rho^{T_c}_{abc}=
                \left[
                \begin{array}{cccccccc}
                0.0 & 0.0 & 0.0 & 0.333 & 0.0 & 0.333 & 0.0 & 0.0 \\
                0.0 & 0.333 & 0.0 & 0.0 & 0.0 & 0.0 & 0.0 & 0.0 \\
                0.0 & 0.0 & 0.333 & 0.0 & 0.333 & 0.0 & 0.0 & 0.0 \\
                0.333 & 0.0 & 0.0 & 0.0 & 0.0 & 0.0 & 0.0 & 0.0 \\
                0.0 & 0.0 & 0.333 & 0.0 & 0.333 & 0.0 & 0.0 & 0.0 \\
                0.333 & 0.0 & 0.0 & 0.0 & 0.0 & 0.0 & 0.0 & 0.0 \\
                0.0 & 0.0 & 0.0 & 0.0 & 0.0 & 0.0 & 0.0 & 0.0 \\
                0.0 & 0.0 & 0.0 & 0.0 & 0.0 & 0.0 & 0.0 & 0.0 \\
                \end{array}
                \right]
                \end{equation}
               All the three matrices have the same set of eigenvalues $-0.471, 0.0, 0.333, 0.471, 0.666$, one of which is negative, thus denoting the presence of entanglement. The reduced density matrix in each case is the same and is given by:
               \begin{equation}
                \rho_{reduced} =
                \left[
                \begin{array}{cccc}
                0.333 & 0.0 & 0.0 & 0.333 \\
                0.0 & 0.333 & 0.0 & 0.0 \\
                0.0 & 0.0 & 0.333 & 0.0 \\
                0.333 & 0.0 & 0.0 & 0.0 \\
                \end{array}
                \right]
                \end{equation}
                The partial transpose of this is the same as that of the matrix. The eigenvalues of the PPT matrix are $-0.206, 0.333, 0.333, 0.539$, one of which is negative, thus denoting the present of entanglement. This verifies the behaviour of the link $ab+ac+bc$. \\[0.3cm]
From the algorithm, the mixed state can be obtained using the state:
$$\ket{\psi_4} = \ket{2^1}_{ab}\ket{0}_c\ket{0}_d+\ket{2^1}_{ac}\ket{1}_b\ket{1}_d+\ket{2^1}_{bc}\ket{0}_a\ket{2}_d$$
Using the above state after tracing out subsystem $d$, we obtained the density matrix as:
\begin{equation}
    \rho_{abc}=
    \left[
    \begin{array}{cccccccc}
    0.25 & 0.0 & 0.0 & 0.125 & 0.0 & 0.0 & 0.125 & 0.0 \\
    0.0 & 0.0 & 0.0 & 0.0 & 0.0 & 0.0 & 0.0 & 0.0 \\
    0.0 & 0.0 & 0.25 & 0.0 & 0.0 & 0.0 & 0.0 & 0.25 \\
    0.125 & 0.0 & 0.0 & 0.125 & 0.0 & 0.0 & 0.0 & 0.0 \\
    0.0 & 0.0 & 0.0 & 0.0 & 0.0 & 0.0 & 0.0 & 0.0 \\
    0.0 & 0.0 & 0.0 & 0.0 & 0.0 & 0.0 & 0.0 & 0.0 \\
    0.125 & 0.0 & 0.0 & 0.0 & 0.0 & 0.0 & 0.125 & 0.0 \\
    0.0 & 0.0 & 0.25 & 0.0 & 0.0 & 0.0 & 0.0 & 0.25 \\
    \end{array}
    \right]
    \end{equation}
    Let us now check the partial transpose for the entire system. The PPT matrices are found to be:\\
    \begin{align*}
        \rho^{T_a}_{abc} &=
        \left[
        \begin{array}{cccccccc}
        0.25 & 0.0 & 0.0 & 0.125 & 0.0 & 0.0 & 0.0 & 0.0 \\
        0.0 & 0.0 & 0.0 & 0.0 & 0.0 & 0.0 & 0.0 & 0.0 \\
        0.0 & 0.0 & 0.25 & 0.0 & 0.125 & 0.0 & 0.0 & 0.0 \\
        0.125 & 0.0 & 0.0 & 0.125 & 0.0 & 0.0 & 0.25 & 0.0 \\
        0.0 & 0.0 & 0.125 & 0.0 & 0.0 & 0.0 & 0.0 & 0.0 \\
        0.0 & 0.0 & 0.0 & 0.0 & 0.0 & 0.0 & 0.0 & 0.0 \\
        0.0 & 0.0 & 0.0 & 0.25 & 0.0 & 0.0 & 0.125 & 0.0 \\
        0.0 & 0.0 & 0.0 & 0.0 & 0.0 & 0.0 & 0.0 & 0.25 \\
        \end{array}
        \right]   \\
        \rho^{T_b}_{abc} &=
        \left[
        \begin{array}{cccccccc}
        0.25 & 0.0 & 0.0 & 0.0 & 0.0 & 0.0 & 0.0 & 0.0 \\
        0.0 & 0.0 & 0.125 & 0.0 & 0.0 & 0.0 & 0.0 & 0.0 \\
        0.0 & 0.125 & 0.25 & 0.0 & 0.125 & 0.0 & 0.0 & 0.25 \\
        0.0 & 0.0 & 0.0 & 0.125 & 0.0 & 0.0 & 0.0 & 0.0 \\
        0.0 & 0.0 & 0.125 & 0.0 & 0.0 & 0.0 & 0.0 & 0.0 \\
        0.0 & 0.0 & 0.0 & 0.0 & 0.0 & 0.0 & 0.0 & 0.0 \\
        0.0 & 0.0 & 0.0 & 0.0 & 0.0 & 0.0 & 0.125 & 0.0 \\
        0.0 & 0.0 & 0.25 & 0.0 & 0.0 & 0.0 & 0.0 & 0.25 \\
        \end{array}
        \right]\\
        \rho^{T_c}_{abc} &=
        \left[
        \begin{array}{cccccccc}
        0.25 & 0.0 & 0.0 & 0.0 & 0.0 & 0.0 & 0.125 & 0.0 \\
        0.0 & 0.0 & 0.125 & 0.0 & 0.0 & 0.0 & 0.0 & 0.0 \\
        0.0 & 0.125 & 0.25 & 0.0 & 0.0 & 0.0 & 0.0 & 0.0 \\
        0.0 & 0.0 & 0.0 & 0.125 & 0.0 & 0.0 & 0.25 & 0.0 \\
        0.0 & 0.0 & 0.0 & 0.0 & 0.0 & 0.0 & 0.0 & 0.0 \\
        0.0 & 0.0 & 0.0 & 0.0 & 0.0 & 0.0 & 0.0 & 0.0 \\
        0.125 & 0.0 & 0.0 & 0.25 & 0.0 & 0.0 & 0.125 & 0.0 \\
        0.0 & 0.0 & 0.0 & 0.0 & 0.0 & 0.0 & 0.0 & 0.25 \\
        \end{array}
        \right]
        \end{align*}
        The set of eigenvalues of $\rho^{T_a}_{abc}$ and $\rho^{T_c}_{abc}$ are the same, namely $\mathbf{-0.146}, \mathbf{-0.052}, 0.0, 0.0, 0.222, 0.25, 0.302, 0.424$ while the eigenvalues of $\rho^{T_b}_{abc}$ are $\mathbf{-0.138}, 0.0, 0.0, 0.107, 0.125, 0.125, 0.25, 0.531$. We see that there exists atleast one negative eigenvalue in all the three matrices, thus denoting the presence of tripartite entanglement. Let us now check the reduced density matrices. 

        \begin{align*}
            \rho_{bc} &=
            \left[
            \begin{array}{cccc}
            0.25 & 0.0 & 0.0 & 0.125 \\
            0.0 & 0.0 & 0.0 & 0.0 \\
            0.0 & 0.0 & 0.375 & 0.0 \\
            0.125 & 0.0 & 0.0 & 0.375 \\
            \end{array}
            \right]\quad \rho_{bc}^{T_b} = 
            \left[
                \begin{array}{cccc}
                0.25 & 0.0 & 0.0 & 0.0 \\
                0.0 & 0.0 & 0.125 & 0.0 \\
                0.0 & 0.125 & 0.375 & 0.0 \\
                0.0 & 0.0 & 0.0 & 0.375 \\
                \end{array}
                \right]
                \\
            \rho_{ac} &=
            \left[
            \begin{array}{cccc}
            0.5 & 0.0 & 0.0 & 0.25 \\
            0.0 & 0.125 & 0.0 & 0.0 \\
            0.0 & 0.0 & 0.125 & 0.0 \\
            0.25 & 0.0 & 0.0 & 0.25 \\
            \end{array}
            \right]\quad \rho_{ac}^{Ta}= \left[
                \begin{array}{cccc}
                0.5 & 0.0 & 0.0 & 0.0 \\
                0.0 & 0.125 & 0.25 & 0.0 \\
                0.0 & 0.25 & 0.125 & 0.0 \\
                0.0 & 0.0 & 0.0 & 0.25 \\
                \end{array}
                \right]\\
            \rho_{ab} &=
            \left[
            \begin{array}{cccc}
            0.25 & 0.0 & 0.0 & 0.125 \\
            0.0 & 0.375 & 0.0 & 0.0 \\
            0.0 & 0.0 & 0.0 & 0.0 \\
            0.125 & 0.0 & 0.0 & 0.375 \\
            \end{array}
            \right]\quad \rho_{ab}^{T_a} = \left[
                \begin{array}{cccc}
                0.25 & 0.0 & 0.0 & 0.0 \\
                0.0 & 0.375 & 0.125 & 0.0 \\
                0.0 & 0.125 & 0.0 & 0.0 \\
                0.0 & 0.0 & 0.0 & 0.375 \\
                \end{array}
                \right]
            \end{align*}
           The set of eigenvalues of PPT matrices of $\rho_{bc}$ and $\rho_{ab}$ are the same given by $\mathbf{-0.038}, 0.25, 0.375, 0.413$ while the eigenvalues of PPT matrix of $\rho_{ac}$ are $\mathbf{-0.125}, 0.25, 0.375, 0.5$. Since in each set, a negative eigenvalue exists, we can conclude that the system exhibits entanglement, thus verifying the behaviour of the link $ab+ac+bc$, that is, even after cutting one variable, the other two remain entangled.
            





\subsection{Applying to Four Qubit Systems}
\section{Physical Significance: Use in Quantum Networks}
\section{Discussion and Conclusion}
\newpage
\section*{Appendix A: {\huge Quantum Information Basics}}
\addcontentsline{toc}{section}{Appendix A: Quantum Information Basics} % Optional
\subsection{Density Matrix}
\subsection{Peres-Horodecki Criterion}
\newpage
\section*{Appendix B: {\huge Knot Theory Basics}}
\addcontentsline{toc}{section}{Appendix B: { Knot Theory Basics}}
\newpage

\section*{Appendix C: {\huge Code for Numeric Calculations}}
We used the \texttt{QuantumInformation.jl} package in \texttt{Julia} to perform the numerical calculations. The code provided below shows some basic calculations that we had used in this report. 
\addcontentsline{toc}{section}{Appendix C: { Code for Numerical Calculations}}
\begin{lstlisting}[language=Julia,label={lst:transmission}]
using QuantumInformation, LinearAlgebra, Latexify
\end{lstlisting}
\bibliographystyle{plain}
\newpage
\bibliography{ref}
\end{document}
% References 
