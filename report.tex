\documentclass{scrartcl}
\usepackage{pgfplots}
\usepackage{makecell}
\usepackage{multirow} % For merged cells
\usepackage{booktabs} % For better table formatting
\usepackage{calc}
\usepackage{Style_File}
\usepackage{fancyhdr}
\usepackage{array}
\newcolumntype{P}[1]{>{\centering\arraybackslash}p{#1}}
% Recommended preamble:
\usetikzlibrary{arrows.meta}
\usetikzlibrary{backgrounds}
\usepgfplotslibrary{patchplots}
\usepgfplotslibrary{fillbetween}
\pgfplotsset{%
    layers/standard/.define layer set={%
        background,axis background,axis grid,axis ticks,axis lines,axis tick labels,pre main,main,axis descriptions,axis foreground%
    }{
        grid style={/pgfplots/on layer=axis grid},%
        tick style={/pgfplots/on layer=axis ticks},%
        axis line style={/pgfplots/on layer=axis lines},%
        label style={/pgfplots/on layer=axis descriptions},%
        legend style={/pgfplots/on layer=axis descriptions},%
        title style={/pgfplots/on layer=axis descriptions},%
        colorbar style={/pgfplots/on layer=axis descriptions},%
        ticklabel style={/pgfplots/on layer=axis tick labels},%
        axis background@ style={/pgfplots/on layer=axis background},%
        3d box foreground style={/pgfplots/on layer=axis foreground},%
    },
}



\setlength{\headheight}{0.75in}
\setlength{\oddsidemargin}{0in}
        \setlength{\evensidemargin}{0in}
        \setlength{\textwidth}{6.5in}
        \setlength{\headwidth}{7.3in}
        \setlength{\textheight}{8.75in}
        \rfoot{\thepage}
        \renewcommand{\headrulewidth}{0pt} % Remove the header line
        \renewcommand{\footrulewidth}{0pt}
\fancyhead[L,C]{}
\fancyhead[L]{PH3203: Term Project}
\fancyhead[R]{0\thepage}
\usepackage[hidelinks]{hyperref}
\hypersetup{colorlinks=true,linkcolor=cyan!80!black}
\fancyhead[C]{ Entanglement Classification using Knots}
\fancyfoot[C]{0\thepage}
\fancyfoot[R,L]{}
\pagestyle{fancy}
\renewcommand{\headrulewidth}{0.4pt}

\usepackage{longtable} 
\usepackage[left = 0.7in,
right = 0.7in,
bottom = 0.9in,
top = 0.9in,
a4paper]{geometry}

\title{
        \Huge\textbf{\textcolor{blue}{{Classification of Entanglement using Knots}}} \\[0.5cm]
        \Large\textsc{{PH3203 Term Project} }
}


\author{{\Large Sagnik Seth} \\ \small{22MS026} \and {\Large Jessica  Das} \\ \small{22MS157} \and {\Large Sayan Karmakar} \\ \small{22MS163}}
\date{}
\begin{document}
\maketitle  
\tableofcontents
\newpage
\section{Introduction}
\section{Classification of Links: A Polynomial Approach}
\section{Entanglement Classification}
\subsection{Obtaining a link from quantum state}
\subsection{Obtaining a state from a link}
In this section, we will see how we can obtain a link from a quantum state. This is in general a difficult task to obtain an entangled quantum states from the polynomial as the number of qubits increases. The process in the paper mentions an algorithm which provides an `incomplete' map betwee a given link and a quantum state. Using the procedure, the general structure of the quantum state can be obtained, however, some free coefficients remain which needs to be fixed computationally. Moreover, presently only mixed states satisfying the link can be obtained using this procedure. \\[0.3cm]
Note that although incomplete, the map is still useful since we can ascertain the structure of each state contained in the mixed state. That is, from a possibility of $2^N$ (for $N$ qubits, there are $2^N$ basis states, namely $\ket{0}, \ket{1}, \ldots \ket{2^N-1}$ where each ket is to be assumed in the binary representation) states, we are reducing it to a much smaller number. \\[0.3cm]
For this, we will use the GHZ type of state as a building block which are of the form:
\begin{align*}
    \ket{N^1} = \frac{1}{\sqrt{2}}\brac{\ket{0}^{\otimes N} + \ket{1}^{\otimes N}} \\
\end{align*}
Here $\ket{0}^{\otimes N}$ is the tensor product of $N$ number of $\ket{0}$ states, that is, $\ket{0}^{\otimes N} = \ket{\underbrace{0000\ldots0}_{\mathrm{N\ times}}}$. The state $\ket{N^1}$ is a maximally entangled state of $N$ qubits. The general algorithm to obtain a state from the link is as follows:
\begin{enumerate}
    \item Let a polynomial $P$ be given. Select a term of the given `link' polynomial, say $t$. 
    \item The term $t$ is then mapped to a state of the form $\ket{\mathrm{E_q}}\otimes\ket{\mathrm{S_q}}\otimes\ket{\mathrm{Q_d}}$ where: 
    \begin{itemize}
        \item $\ket{\mathrm{E_q}}$ is the entangled qubit of the GHZ type as specified above, associated to ring variables contained in $t$.
        \item $\ket{\mathrm{S_q}}$ is a separable qubit associated with ring variables not contained in $t$. There are a number of possibilities for this separable qubit and we have to find it computationally.
        \item $\ket{\mathrm{Q_d}}$ is a qudit state which is associated with an artifically introduced ring variable (which is alphabetically the next letter of the largest ring variable present). The states always starts from 0 for the first term and is increased by 1 for each successive term of the polynomial. This will later be traced out, hence is of less significance. 
    \end{itemize}
    \item The full state $\ket{\psi}$ is constructed by summing these individual states obtained for each term of the polynomial. 
    \item The full mixed state characterised by this polynomial is then obtained by tracing out the qudit state $\ket{\mathrm{Q_d}}$.
    $$\boxed{\hat{\rho}(P)=\frac{\Tr_d{\ket{\psi}\bra{\psi}}}{\sqrt{\braket{\psi|\psi}}}}$$
\end{enumerate}
\textbf{Example demonstrating the algorithm:} \\[0.3cm]
We will see a simple example of the algorithm to obtain a state from a link. Consider the polynomial $P(a,b,c) = ab+ac$. This is a three-ring link. 
\begin{itemize}
    \item Let us choose the term $t = ab$. This term has two ring variables thus we will associate a two qubit GHZ type of state to $\ket{\mathrm{E_q}}$. Thus, we have $\ket{\mathrm{E_q}} = \frac{1}{\sqrt{2}}\brac{\ket{00} + \ket{11}}\equiv \ket{2^1}_{ab}$. \\[0.3cm]Since the separable qubit has large possibility, we will denote it generally by $\ket{q_1}$ and this will be associated with the remaining ring variable which is $c$. Thus, $\ket{\mathrm{S_q}} = \ket{q_1}_c$. The remaining term is the qudit state which will be associated to $d$ (since $d$ is alphabetical successor of the largest ring variable $c$). Then we will have the full state:
\begin{align*}
    \ket{\psi_1} &= \ket{2^1}_{ab}\otimes\ket{q_1}_c\otimes\ket{0}_d 
\end{align*}
\item Now, let us choose the next term in the polynomial which is $t = ac$. Similar to above, to the entangled qubit we will associate the two qubit GHZ state, thus, $\ket{\mathrm{E_q}} = \frac{1}{\sqrt{2}}\brac{\ket{00} + \ket{11}}\equiv \ket{2^1}_{ac}$.\\[0.3cm] The separable qubit will be associated with the remaining ring variable $b$ and we will denote it by $\ket{q_2}_b$. The qudit state will be associated with $d$ which is the alphabetical successor of $c$ but this time we will use $\ket{1}_d$ as for each successive term, the qudit state increases to the next level. Thus, we have:
\begin{align*}
    \ket{\psi_2} &= \ket{2^1}_{ac}\otimes\ket{q_2}_b\otimes\ket{1}_d
\end{align*}
\item The full state $\ket{\psi}$ is then obtained by summing the two states obtained above with some coefficients:
\begin{align*}
    \ket{\psi} &= c_1\ket{\psi_1} + c_2\ket{\psi_2} \\
    &=c_1(\ket{2^1}_{ab}\otimes\ket{q_1}_c\otimes\ket{0}_d )+ c_2(\ket{2^1}_{ac}\otimes\ket{q_2}_b\otimes\ket{1}_d)
\end{align*}
Then we can trace out the qudit state $\ket{d}$ to obtain the density matrix of the ring variables:
$$\boxed{\hat{\rho}_{abc}=\frac{\Tr_d{\ket{\psi}\bra{\psi}}}{\sqrt{\braket{\psi|\psi}}}} $$
\end{itemize}
\subsection{Applying to Three Qubit Systems}
\subsection{Applying to Four Qubit Systems}
\section{Physical Significance: Use in Quantum Networks}
\section{Discussion and Conclusion}
\newpage
\section*{Appendix A: {\huge Quantum Information Basics}}
\addcontentsline{toc}{section}{Appendix A: Quantum Information Basics} % Optional
\subsection{Density Matrix}
\subsection{Peres-Horodecki Criterion}
\newpage
\section*{Appendix B: {\huge Knot Theory Basics}}
\addcontentsline{toc}{section}{Appendix B: { Knot Theory Basics}}
\end{document}