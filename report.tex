\documentclass{scrartcl}
\usepackage{pgfplots}
\usepackage{makecell}
\usepackage{multirow} % For merged cells
\usepackage{booktabs} % For better table formatting
\usepackage{calc}
\usepackage{Style_File}
\usepackage{fancyhdr}
\usepackage{array}
\newcolumntype{P}[1]{>{\centering\arraybackslash}p{#1}}
% Recommended preamble:
\usetikzlibrary{arrows.meta}
\usetikzlibrary{backgrounds}
\usepgfplotslibrary{patchplots}
\usepgfplotslibrary{fillbetween}
\pgfplotsset{%
    layers/standard/.define layer set={%
        background,axis background,axis grid,axis ticks,axis lines,axis tick labels,pre main,main,axis descriptions,axis foreground%
    }{
        grid style={/pgfplots/on layer=axis grid},%
        tick style={/pgfplots/on layer=axis ticks},%
        axis line style={/pgfplots/on layer=axis lines},%
        label style={/pgfplots/on layer=axis descriptions},%
        legend style={/pgfplots/on layer=axis descriptions},%
        title style={/pgfplots/on layer=axis descriptions},%
        colorbar style={/pgfplots/on layer=axis descriptions},%
        ticklabel style={/pgfplots/on layer=axis tick labels},%
        axis background@ style={/pgfplots/on layer=axis background},%
        3d box foreground style={/pgfplots/on layer=axis foreground},%
    },
}



\setlength{\headheight}{0.75in}
\setlength{\oddsidemargin}{0in}
        \setlength{\evensidemargin}{0in}
        \setlength{\textwidth}{6.5in}
        \setlength{\headwidth}{7.3in}
        \setlength{\textheight}{8.75in}
        \rfoot{\thepage}
        \renewcommand{\headrulewidth}{0pt} % Remove the header line
        \renewcommand{\footrulewidth}{0pt}
\fancyhead[L,C]{}
\fancyhead[L]{PH3203: Term Project}
\fancyhead[R]{0\thepage}
\usepackage[hidelinks]{hyperref}
\hypersetup{colorlinks=true,linkcolor=cyan!80!black}
\fancyhead[C]{ Entanglement Classification using Knots}
\fancyfoot[C]{0\thepage}
\fancyfoot[R,L]{}
\pagestyle{fancy}
\renewcommand{\headrulewidth}{0.4pt}

\usepackage{longtable} 
\usepackage[left = 0.7in,
right = 0.7in,
bottom = 0.9in,
top = 0.9in,
a4paper]{geometry}

\title{
        \Huge\textbf{\textcolor{blue}{{Classification of Entanglement using Knots}}} \\[0.5cm]
        \Large\textsc{{PH3203 Term Project} }
}


\author{{\Large Sagnik Seth} \\ \small{22MS026} \and {\Large Jessica  Das} \\ \small{22MS157} \and {\Large Sayan Karmakar} \\ \small{22MS163}}
\date{}
\begin{document}
\maketitle  
\tableofcontents
\newpage
\section{Introduction}
%     Source - Wikipedia

Classifying entanglement is essential because not all quantum states are equally useful for quantum information tasks. Different types of entanglement serve as distinct resources, each suited to specific applications such as quantum algorithms or secure communication protocols like quantum key distribution. Understanding and identifying these entanglement types helps determine how quantum states can be used and manipulated effectively. hi hi hi this is testing


SLOCC (Stochastic Local Operations and Classical Communication) is a method for classifying quantum entanglement. It defines equivalence classes of quantum states based on whether they can be converted into each other using local operations (on individual qubits) and classical communication. 

This idea has been used successfully to study three-qubit states, as shown in [1,2], classifying four-qubit states [3–6] Methods have been developed for handling systems with even more qubits [7,8].

In this paper, the authors have proposed an alternative classification scheme for quantum entanglement based on topological links.


One of the first images that comes to mind when we think of entanglement is that of entangled threads. Naturally, one wonders if we could study quantum entanglement using entangled 'knots'. Aravind~\cite{Aravind} was the first to point out the connections between entangled quantum states and classical knot configurations, focussing on similarity between 3-particle GHZ state and Borromean rings. He associated each particle with a ring, entanglement of any set of particles as inability to separate their corresponding rings, and measurement of particle state as cutting its ring. But he noted that performing the measurement in different basis would not lead to the same conclusions. This limit in analogy was dealt with by Sugita~\cite{Sugita}. He proposed that cutting the ring is equivalent to tracing out the corresponding particle from the density operator, which is a basis-independent operation. This represents viewing the system as though that particle is no longer present. Moreover, the trace operation helps to generalise the idea to quantum systems with more than 2 levels.  



\section{Classification of Links: A Polynomial Approach}
\section{Entanglement Classification}
\subsection{Obtaining a link from quantum state}
\subsection{Obtaining a state from a link}

HI  I am doing this on FRIDAY 


\subsection{Applying to Three Qubit Systems}
\subsection{Applying to Four Qubit Systems}
\section{Physical Significance: Use in Quantum Networks}
\section{Discussion and Conclusion}
% References 
\begin{thebibliography}{99}

    \bibitem{Aravind}
    P. K. Aravind, Borromean entanglement of the GHZ state,
     in \emph{Quantum Potentiality, Entanglement and Passion-at-a
    Distance: Essays for Abner Shimony}, editedbyR.S.Cohen,
     M. Horne, and J. Stachel (Kluwer, Dordrecht, 1997), p. 53.
    \bibitem{Sugita}
    A. Sugita, Borromean entanglement revisited, in \emph{Proceedings of
     the International Workshop on Knot Theory for Scientic Objects}
     (Osaka, Japan, 2006).
    
    \end{thebibliography}
\newpage
\section*{Appendix A: {\huge Quantum Information Basics}}
\addcontentsline{toc}{section}{Appendix A: Quantum Information Basics} % Optional
\subsection{Density Matrix}
\subsection{Peres-Horodecki Criterion}
\newpage
\section*{Appendix B: {\huge Knot Theory Basics}}
\addcontentsline{toc}{section}{Appendix B: { Knot Theory Basics}}
\end{document}